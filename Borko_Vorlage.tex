\documentclass[a4paper,12pt]{scrreprt}
\usepackage[T1]{fontenc}
\usepackage[utf8]{inputenc}
\usepackage[ngerman]{babel}
\usepackage[table]{xcolor}% http://ctan.org/pkg/xcolor
\usepackage{tabu}
\usepackage{graphicx}
\usepackage{lmodern}

\begin{document}


%\titlehead{Kopf} %Optionale Kopfzeile
\author{Daniel Dimtrijevic \and Thomas Traxler} %Zwei Autoren
\title{ Verschl"usselung } %Titel/Thema
\subject{VSDB} %Fach
\subtitle{ DES \\ Diffie-Hellman } %Genaueres Thema, Optional
\date{\today} %Datum
\publishers{5AHITT} %Klasse

\maketitle
\tableofcontents


\chapter{Aufgabenstellung}
Kommunikation [12Pkt]
Programmieren Sie eine Kommunikationsschnittstelle zwischen zwei Programmen (Sockets; Übertragung von Strings). Implementieren Sie dabei eine unsichere (plainText) und eine sichere (secure-connection) Übertragung.

Bei der secure-connection sollen Sie eine hybride Übertragung nachbilden. D.h. generieren Sie auf einer Seite einen privaten sowie einen öffentlichen Schlüssel, die zur Sessionkey Generierung verwendet werden. Übertragen Sie den öffentlichen Schlüssel auf die andere Seite, wo ein gemeinsamer Schlüssel für eine synchrone Verschlüsselung erzeugt wird. Der gemeinsame Schlüssel wird mit dem öffentlichen Schlüssel verschlüsselt und übertragen. Die andere Seite kann mit Hilfe des privaten Schlüssels die Nachricht entschlüsseln und erhält den gemeinsamen Schlüssel.

Sniffer [4Pkt]
Schreiben Sie ein Sniffer-Programm (Bsp. mithilfe der jpcap-Library http://jpcap.sourceforge.net  oder jNetPcap-Library http://jnetpcap.com/), welches die plainText-Übertragung abfangen und in einer Datei speichern kann. Versuchen Sie mit diesem Sniffer ebenfalls die secure-connection anzuzeigen.

Info
Gruppengröße: 2 Mitglieder
Punkte: 16

    Erzeugen von Schlüsseln: 4 Punkte
    Verschlüsselte Übertragung: 4 Punkte
    Entschlüsseln der Nachricht: 4 Punkte
    Sniffer: 4 Punkte

	
\chapter{Designüberlegungen}
	
\chapter{Arbeitsaufteilung}
	\tabulinesep = 4pt
	\begin{tabu}  {|[2pt]X[2.5,c] |[1pt] X[4,c] |[1pt]X[1.3,c]|[1pt]X[c]|[2pt]}
		\tabucline[2pt]{-}
		Name & Arbeitssegment & Time Estimated & Time Spent\\\tabucline[2pt]{-}
		
		Thomas Traxler & Chat & 1h & 0.5h\\\tabucline[1pt]{-}
		Daniel Dimitrijevic & En-/Decrypt & 2h & 2h\\\tabucline[1pt]{-}
		Thomas Traxler & En-/Decrypt & 1h & 1h\\\tabucline[1pt]{-}
		Walter Klaus & Diffie-Hellman & 1h & 2h\\\tabucline[1pt]{-}
		Daniel Dimitrijevic & Sniffing & 1h & 2h\\\tabucline[2pt]{-}
		Gesamt && 6h & 7.5h\\\tabucline[2pt]{-}
	\end{tabu}	
\chapter{Arbeitsdurchführung}

Beim Sniffer habe ich mir die Vorlagen von Jnetpcap angeschaut und diese zur Kreation meines Sniffers herangenommen.
Im der ersten Phase des Sniffers suche ich mir die Liste an Adaptern heraus die auf die der Sniffer sniffen kann.
Das hat am Anfang zu Problemen bei mir gef"uhrt da ich noch von den Vorjahren einige Loopback adapter hatte die ich erst deaktivieren musste. Nachdem ich dieses Problem gel"ost hatte ging ich die Aufgabe der Packetausgabe an. Da hatte ich f"ur die normale Ausgabe keine Problem, bis ich bemerkte das ich Limitieren musste welche Packete mitgesnifft werden m"ussen. Nach einer weile Recherche habe ich auch dieses Problem gel"ost. Danach war die Ausgabe der Packete nur noch ein kleines Problem. Es kostet mich ein wenig extra Zeit herauszufinden mit welchem Befehl ich die Payload eines Packetes ausgeben kann und es in Hexadezimal auch anzuzeigen. Danach war nur noch die Anzahl der Durchläufe einzugeben und den Sniffer zuschlie"sen. 
\chapter{Testbericht}
\chapter{Quellen}

\end{document}